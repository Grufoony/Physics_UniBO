\documentclass[12pt]{book}
\usepackage{hyperref}

\title{Formulary for Statistical Data Analysis} \author{\url{https://github.com/Grufoony/Physics_Unibo}}
\date{}

\usepackage{amsmath}
\usepackage{amsfonts}
\usepackage{amssymb}
\usepackage{amsthm}
\usepackage{braket}
\usepackage[margin=3cm]{geometry}
\usepackage{pgfplots}
\pgfplotsset{compat=1.18}
\usepackage{fancyhdr}
\usepackage{physics}
\usepackage{systeme,mathtools}
\usepackage{graphicx}
\usepackage{float}
\usepackage{relsize}
\usepackage{calligra}
\usepackage{siunitx}
\usepackage{circuitikz}
\usepackage[miktex]{gnuplottex}
\usepackage{epstopdf}
\usepackage[english]{babel}
\usepackage{float}
\usepackage{tikz}


\newcommand{\vv}{\vec{v}}
\newcommand{\vw}{\vec{w}}
\newcommand{\vo}{\vec{0}}
\newcommand{\vx}{\vec{x}}
\newcommand{\R}{\Re}
\newcommand{\la}{\lambda}
\newcommand{\bd}{\textbf}
\newcommand{\lang}{\left\langle}
\newcommand{\rang}{\right\rangle}
\newcommand{\lbra}{\left\lbrace}
\newcommand{\rbra}{\right\rbrace}
\newcommand{\ih}{\hat{i}}
\newcommand{\jh}{\hat{j}}
\newcommand{\kh}{\hat{k}}
\newcommand{\vr}{\vec{r}}

\begin{document}

\maketitle
\tableofcontents
\pagebreak


\chapter{Probability theory}
\section{Combinatorics}
Permutations without repetitions:
\begin{equation}
	P_{n} = n!
\end{equation}
Permutations with repetitions:
\begin{equation}
	P^r_{n} = \frac{n!}{\prod k_i!}
\end{equation}
Dispositions without repetitions:
\begin{equation}
	D_{n,k} = \frac{n!}{(n-k)!}
\end{equation}
Dispositions with repetitions:
\begin{equation}
	D^r_{n,k} = n^k
\end{equation}
Combinations without repetitions:
\begin{equation}
	C_{n,k} = \begin{pmatrix}
		n \\
		k
	\end{pmatrix} = \frac{n!}{k!(n-k)!}
\end{equation}
Combinations with repetitions:
\begin{equation}
	C^r_{n,k} = \begin{pmatrix}
		n+k-1 \\
		k
	\end{pmatrix}
\end{equation}
\section{Probability}
Conditional probability:
\begin{equation}
	P(A|B) = \frac{P(A\cap B)}{p(B)}
\end{equation}
Probability of intersection for independent events:
\begin{equation}
	P(A\cap B) = P(A)P(B)
\end{equation}
Bayes' theorem:
\begin{equation}
	P(A|B) = \frac{P(B|A)P(A)}{P(B)}
\end{equation}
Law of total probability:
\begin{equation}
	P(A|B) = \frac{P(B|A)P(A)}{\sum_i P(B|A_i)P(A_i)}
\end{equation}
Bayes' theorem in Baeysian thinking:
\begin{equation}
	P(H|\vx) = \frac{P(\vx|H)\pi(H)}{\int P(\vx|H)\pi(H)dH}
\end{equation}
\section{Random variables and distributions}
Marginal pdf:
\begin{equation}
	f_i(x_i) = \int f(\vx)\prod_j dx_j
\end{equation}
Conditional pdfs:
\begin{equation}
	f(y|x) = \frac{f(x,y)}{f_x(x)} f(x|y) = \frac{f(x,y)}{f_y(y)}
\end{equation}
Bayes' theorem for distributions
\begin{equation}
	f(x|y) = \frac{f(y|x)f_x(x)}{f_y(y)}
\end{equation}
Condition for independent variables:
\begin{equation}
	f(x,y) = f_x(x)f_y(y)
\end{equation}
Distribution of a function of a random variable in 1-D:
\begin{equation}
	g(a) = f(x(a))\vert\frac{dx}{da}\vert
\end{equation}
Distribution of a function of a random variable in N-D:
\begin{equation}
	g(\vec{y}) = |J|f(\vx)
\end{equation}
Expectation value:
\begin{equation}
	E[x] = \int xf(x)dx = \mu_x
\end{equation}
Variance:
\begin{equation}
	V[x] = E[x^2] - E[x]^2
\end{equation}
Covariance:
\begin{equation}
	cov[x,y] = E[xy] - E[x]E[y] = E[(x-\mu_x)(y-\mu_y)]
\end{equation}
Correlation coefficient:
\begin{equation}
	\rho_{xy} = \frac{cov[xy]}{\sigma_x\sigma_y}
\end{equation}
Correlation matrix:
\begin{equation}
	V_{ij} = cov[x_i,x_j]
\end{equation}
Variance of a function of random variables:
\begin{equation}
	\sigma^2_y \approx \sum_{i,j} \left[\frac{\partial y}{\partial x_i}\frac{\partial y}{\partial x_j}\right]_{\vx=\vec{\mu}} V_{ij}
\end{equation}
Variance of a vector function of random variables:
\begin{equation}
	U_{kl} \approx \sum_{i,j} \left[\frac{\partial y_k}{\partial x_i}\frac{\partial y_k}{\partial x_j}\right]_{\vx = \vec{\mu}} V_{ij}
\end{equation}
Error propagation for sum of uncorrelated variables:
\begin{equation}
	\sigma^2_y = \sigma^2_1 + \sigma^2_2 +2cov[x_1,x_2]
\end{equation}
Error propagation for product of uncorrelated variables:
\begin{equation}
	\frac{\sigma^2_y}{y^2} = \frac{\sigma^2_1}{x^2_1} + \frac{\sigma^2_2}{x^2_2} + 2\frac{cov[x_1,x_2]}{x_1x_2}
\end{equation}
Characteristic function:
\begin{equation}
	\phi_x(k) = E[e^{ikx}] = \int_{-\infty}^\infty e^{ikx}f(x)dx
\end{equation}
Moments of Characteristic function:
\begin{equation}
	\frac{d^m}{dk^m}\phi_z(k) = i^m \mu'_m
\end{equation}
\section{Important distributions}
\subsection{Binomial distribution}
Distribution:
\begin{equation}
	f(n;N,p) = \frac{N!}{n!(N-n)!}p^n(1-p)^{N-n}
\end{equation}
Expectation value:
\begin{equation}
	E[n] = Np
\end{equation}
Variance:
\begin{equation}
	V[n] = Np(1-p)
\end{equation}
\subsection{Poisson distribution}
Distribution:
\begin{equation}
	f(n;\nu) = \frac{\nu^n}{n!}e^{\nu}
\end{equation}
Expectation value:
\begin{equation}
	E[n] = \nu
\end{equation}
Variance:
\begin{equation}
	V[n] = \nu
\end{equation}
\subsection{Uniform distribution}
Distribution:
\begin{equation}
	f(x;\alpha,\beta) = 
\end{equation}
Expectation value:
\begin{equation}
	E[x] = \frac{\alpha + \beta}{2}
\end{equation}
Variance:
\begin{equation}
	V[x] = \frac{(\beta - \alpha)^2}{12}
\end{equation}
\subsection{Exponential distribution}
Distribution:
\begin{equation}
	f(x;\xi)
\end{equation}
Expectation value:
\begin{equation}
	E[x] = \xi
\end{equation}
Variance:
\begin{equation}
	V[x] = \xi^2
\end{equation}
\subsection{Gaussian distribution}
Distribution:
\begin{equation}
	f(x;\mu,\sigma)
\end{equation}
Expectation value:
\begin{equation}
	E[x] = \mu
\end{equation}
Variance:
\begin{equation}
	V[x] = \sigma^2
\end{equation}
\subsection{Multivariate Gaussian distribution}
Distribution:
\begin{equation}
	f(\vx;\vec{mu},V)
\end{equation}
\subsection{Chi-square distribution}
Distribution:
\begin{equation}
	f()
\end{equation}
Expectation value:
\begin{equation}
	E[]
\end{equation}
Variance:
\begin{equation}
	V[]
\end{equation}
\subsection{Cauchy distribution}
Distribution:
\begin{equation}
	f()
\end{equation}
\subsection{Student's t distribution}
Distribution:
\begin{equation}
	f()
\end{equation}
Expectation value:
\begin{equation}
	E[x] = 0
\end{equation}
Variance:
\begin{equation}
	V[x] = \frac{\nu}{\nu-2}
\end{equation}

\end{document}
