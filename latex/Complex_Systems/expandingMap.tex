Now let's see an example of a chaotic map.
First of all, we define a map $T$ as \emph{expanding} if $\left\lvert T'\left(x\right)\right\rvert > 1$ $\forall x \in \mathcal{M}$. \\
For example, let's take the map $T : \left[0,1\right] \to \left[0,1\right]$ such as $T\left(x\right)=2x \mod{1}$.
This map as two fixed points (intersection with the bisectrice): $p_0 = \left(0,0\right)$ and $p_1 = \left(1,1\right)$.
We notice immediately that those points are both \emph{repelling} points.
Another thing we can see with this particular map is that if we start on a slightly different point from the previous one, for example $x \to x + \epsilon$, the difference between the two orbit grows exponentially fast:
$$
    T^n\left(x\right)=2^n\left(x-k\left(x\right)\right)-k\left(T\left(x\right)\right)
$$
where
$$
    k\left(x\right)=
    \begin{cases}
        0 \quad 0\leq x \leq \frac{1}{2} \\
        1 \quad \frac{1}{2} < x \leq 1
    \end{cases}
$$
and in conclusion we have that
$$
    T\left(x + \epsilon\right)=2^n\left(x + \epsilon\right)\mod{1}
$$
so the difference grows $\propto 2^n\epsilon$.
One last comment to make is that it's true that the two orbit exponentially diverges one from the other but the phase space is limited: from this point of view their distance in phase space will oscillate through time.


\begin{prop}
    If $T : \left[0,1\right] \to \left[0,1\right]$ is an expanding map such as $\left\lvert T'\left(x\right)\right\rvert > 1$ $\forall x \in \left[0,1\right]$ then any point p of $T^j$ ($j \in \mathbb{Z}$) is repelling.
\end{prop}
\begin{proof}
	If we take the derivative of $T^j$ we get
	$$
		(T^j)'(x) = T'(T^{j-1}(x))\times T'(T^{j-2}(x))\times...\times T'(x) = \prod_{k=0}^{j-1}T'\left(T^k\left(x\right)\right)
   $$
  	and if now we calculate the module of this derivative we get 
   $$
       \left\lvert\left(T^j\right)'\left(x\right)\right\rvert = \prod_{k=0}^{j-1}\left\lvert T'\left(T^k\left(x\right)\right)\right\rvert \gg 1
   $$
	where, since the map is expanding, each term of the product is larger then one.
\end{proof}
\begin{prop}
    If two points p and q are part of the same periodic orbit then they are both of the same type.
\end{prop}
\begin{proof}
    Let's consider an orbit of period $j$ with $q=T^k\left(p\right)$, $0 < k \leq j-1$.
    $$
        \left(T^j\right)'\left(p\right) = T'\left(p_0\right)\ldots T'\left(p_{j-1}\right)
    $$
    $$
        \left(T^j\right)'\left(q\right) = \left(T^j\right)'\left(p_k\right) T'\left(p_k\right)\ldots T'\left(p_{k-1}\right) \ldots T'\left(p_{k+j-1}\right)
    $$
    Real numbers always commute, so it's obvious that
    $$
        \left(T^j\right)'\left(q\right) = \left(T^j\right)'\left(p\right)
    $$
\end{proof}
We have just shown that the property of a point (local) is also a property of an orbit (global).
