This model takes into account the fact that the higher the population is, the more resources are needed, so there should be a limit that can't be surpassed, because the environment wouldn't be able to sustain it
$$
	x_{n+1} = \alpha x_n(A-x_n)
$$
Now we change variable, defining $y = x/A$, and we get
$$
	\frac{x_{n+1}}{A} = A\alpha \frac{x_n}{A}\left(1 - \frac{x_n}{A}\right)
$$
$$
	y_{n+1} = r y_n(1-y_n)
$$
which is the logistic map:
\begin{equation}
	T(x) = rx(1-x)
\end{equation}
We take $0 < x < 1$ and $1 < r < 4$. The graph is a parabola upside down, with the maximum at $\left(\frac{x}{2},\frac{r}{4}\right)$.
It may happen that $T\left(x\right)^n=x$: in this case the orbit $O\left(x\right)=\lbra\Phi^t\left(x\right)\rbra_{x \in \mathbb{G}}$ is called \emph{periodic orbit}.
We also observe that getting closer to an orbit is not the same as getting closer to a limit.