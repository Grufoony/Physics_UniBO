We need to define a mathematical model that generates sequences from an \emph{alphabet} $\mathcal{A}$, which can be any finite set.
We will denote both set of finite and infinite sequences as $\mathcal{A}^{*} = \cup_{n \in \mathbb{N}} \mathcal{A}^n$ and $\mathcal{A}^\mathbb{N}$.
Now we can define a sequence, or a \emph{word}, $\omega \in \mathcal{A}^n$ and denote with $\left\lvert \omega \right\rvert = n$ its length.
In particular, we will use the notation $\omega_i^j = \left(\omega_i, \ldots, \omega_j\right)$.
We can also take $\mathcal{A}^\mathbb{Z}$ as two-sided alphabet.

\begin{definition}
    A \textbf{measurable space} $\left(\Omega, \mathcal{F}\right)$ is (usually) defined by a compact metric space $\Omega$ and a $\sigma$-algebra $\mathcal{F}$.
\end{definition}

We will denote the canonical cylinder on $\Omega$ as $\left[a_1^n\right] = \left\{y \in \Omega \ \vert \ y_1 = x_1, \ldots, y_n = x_n\right\}$.
To figure out that this is actually a cylinder, let's pretend to take $\left(r, \phi, h\right)$ cylindrical coordinates, fixing the radius $r = r_0$, letting the angle and the height free.
In other words, a cylinder contains all sequences which start with a given word.

Our space has a topology, so we can take $\mu \approx$ m (metric) absolutely continuous w.r.t. the Lebesgue measure on $\Omega = \mathbb{R}^n$.
So it exists $\varphi \in \mathbb{L}^1(m)$ such that $\mu(f) = \int dm f(m)\varphi(m)$.
Consider now the function
\begin{equation*}
    g_\mathcal{A}\left(z, z'\right) =
    \begin{cases}
        1 \ \ z = z' \\
        0 \ \ z \neq z'
    \end{cases}
    \ \ \forall z, z' \in \mathcal{A}
\end{equation*}
Taking $x, y \in \Omega$ infinite sequences it is possible to prove that
\begin{equation*}
    \widetilde{d}\left(x, y\right) = \sum_{n=1}^\infty 2^{-n} g_\mathcal{A}\left(x_n, y_n\right)
\end{equation*}
is a metric over $\Omega$.
Taking $x^{(n)} \in \Omega$ sequence of infinite sequences, given $0 < \lambda = \frac{1}{\left\lvert A\right\rvert} < 1$, we have that $d(x, y) = \lambda^{n\left(x, y\right)} \ \ \forall x, y \in \Omega$ is also a metric over $\Omega$, with $n(x, y) = \min\left\{k \vert x_k \neq y_k\right\}$.
Moreover, $d$ and $\widetilde{d}$ define the same topology.
The open balls are, $\forall x \in \Omega, \ \ r > 0$
\begin{equation*}
    \mathcal{B}\left(x, r\right) =  \left\{y \in \Omega \ \vert \ d(x, y) \leq r\right\} = \left\{y \in \Omega \ \vert \ x_k = y_k \ \forall \ 1 \leq k \leq \frac{\ln r}{\ln \lambda}\right\}
\end{equation*}
\begin{definition}
    $\mathcal{F}$ is a \textbf{Borel $\sigma$-algebra} if is a set of subsets of $\Omega$ such that $\Omega \in \mathcal{F}$
\end{definition}
So a $\sigma$-algebra is actually a collection of all measurable sets.
\begin{definition}
    The \textbf{space of probability measures} is $\mathcal{P}\left(\Omega\right) = \left\{\mu \ \vert \ \mu\left(\Omega\right) = 1\right\}$
\end{definition}
Taking now a map $T : \Omega \to \Omega$ with a probability measure $\mu$, $\mu$ is called \textbf{T-invariant} if $\forall A \in \mathcal{F}, \mu\left(T^{-1}\left(A\right)\right) = \mu\left(A\right)$.
In other words, the push-forward measure as to satisfy $T * \mu \equiv \mu \cdot T^{-1} = \mu$.
We will denote the space of invariant probability measures with $\mathcal{P}_\mathcal{F} \subset \mathcal{P}$.

The \textbf{shift} map $\sigma : \Omega \to \Omega$ such that $\sigma\left(x_0, x_1, \ldots, x_n, \ldots\right) = \sigma\left(x_1, \ldots, x_n, \ldots\right)$ represents an important map over our space.
Notice that the shift map could represent our time, helping us to code a dynamical system (more details in the next sections).
Moreover, one can verify that if $\mu$ is i.i.d. then it's shift-invariant, i.e. $\mu\left(\sigma^{-1}\left(\left[x_1^n\right]\right)\right) = \mu\left(\left[x_1^n\right]\right)$ and $\mu\left(\left[x_{n+k}^{m+k}\right]\right) = \mu\left(\left[x_n^m\right]\right)$.

\begin{theorem}[Borel-Cantelli lemma]
    \begin{equation*}
        \left\{E_n\right\} \ \vert \ \sum_{n=1}^\infty \mu\left(E_n\right) < \infty \ \implies \mu\left(\lim \sup E_n\right) = 0
    \end{equation*}
\end{theorem}
This lemma states that, beside null-measure sets, typically an $x \in \Omega$ only belongs to finitely many $E_k$'s. \\
In order to clarify, let's take $E_m \subset \Omega$ such that $x \in \lim \sup E_n \Leftrightarrow x \in \cup_{m=n}^\infty E_m \ \forall n$, then it exists $n_j \to \infty$ sequence such that $x \in E_{n_j} \ \forall j$.
Consequently, $E_n$ are becoming rare as $n$ increases.

\begin{definition}
    Let's consider a sequence of events $\left\{E_1, \ldots, E_n\right\}$.
    These are \textbf{independent} if $\mu\left(E_1 \cap \ldots \cap E_n\right) = \mu\left(E_1\right) \ldots \mu\left(E_n\right)$
\end{definition}