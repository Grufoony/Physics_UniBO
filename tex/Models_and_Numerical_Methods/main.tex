\documentclass[12pt]{article}
\usepackage{hyperref}

\title{Lecture notes from \\ Models and Numerical Methods} \author{\url{https://github.com/Grufoony/Physics_Unibo}}
\date{}

\usepackage{amsmath}
\usepackage{amsfonts}
\usepackage{amssymb}
\usepackage{amsthm}
\usepackage{braket}
\usepackage[margin=3cm]{geometry}
\usepackage{pgfplots}
\pgfplotsset{compat=1.18}
\usepackage{fancyhdr}
\usepackage{physics}
\usepackage{systeme,mathtools}
\usepackage{graphicx}
\usepackage{float}
\usepackage{relsize}
\usepackage{calligra}
\usepackage{siunitx}
\usepackage{circuitikz}
\usepackage[miktex]{gnuplottex}
\usepackage{epstopdf}
\usepackage[english]{babel}
\usepackage{float}
\usepackage{tikz}
\usepackage{pgfplots}
\usetikzlibrary{shapes}

\newtheorem{theorem}{Theorem}
\newtheorem{definition}{Definition}

\begin{document}

\maketitle

\newpage
\thispagestyle{empty}
\addtocounter{page}{-2}
\mbox{}

\tableofcontents
\pagebreak


\section{Resume of measure theory}
We need to define a mathematical model that generates sequences from an \emph{alphabet} $\mathcal{A}$, which can be any finite set.
We will denote both set of finite and infinite sequences as $\mathcal{A}^{*} = \cup_{n \in \mathbb{N}} \mathcal{A}^n$ and $\mathcal{A}^\mathbb{N}$.
Now we can define a sequence, or a \emph{word}, $\omega \in \mathcal{A}^n$ and denote with $\left\lvert \omega \right\rvert = n$ its length.
In particular, we will use the notation $\omega_i^j = \left(\omega_i, \ldots, \omega_j\right)$.
We can also take $\mathcal{A}^\mathbb{Z}$ as two-sided alphabet.

\begin{definition}
    A \textbf{measurable space} $\left(\Omega, \mathcal{F}\right)$ is (usually) defined by a compact metric space $\Omega$ and a $\sigma$-algebra $\mathcal{F}$.
\end{definition}

We will denote the canonical cylinder on $\Omega$ as $\left[a_1^n\right] = \left\{y \in \Omega \ \vert \ y_1 = x_1, \ldots, y_n = x_n\right\}$.
To figure out that this is actually a cylinder, let's pretend to take $\left(r, \phi, h\right)$ cylindrical coordinates, fixing the radius $r = r_0$, letting the angle and the heigh free.

Our space has a topology, so we can take $\mu \approx$ m (metric) absolutely continuous w.r.t. the Lebesgue measure on $\Omega = \mathbb{R}^n$.
So it exists $\varphi \in \mathbb{L}^1(m)$ such that $\mu(f) = \int dm f(m)\varphi(m)$.
Consider now the function
\begin{equation*}
    g_\mathcal{A}\left(z, z'\right) =
    \begin{cases}
        1 \ \ z = z' \\
        0 \ \ z \neq z'
    \end{cases}
    \ \ \forall z, z' \in \mathcal{A}
\end{equation*}
Taking $x, y \in \Omega$ infinite sequences it is possible to prove that
\begin{equation*}
    \widetilde{d}\left(x, y\right) = \sum_{n=1}^\infty 2^{-n} g_\mathcal{A}\left(x_n, y_n\right)
\end{equation*}
is a metric over $\Omega$.
Taking $x^{(n)} \in \Omega$ sequence of infinite sequences, given $0 < \lambda = \frac{1}{\left\lvert A\right\rvert} < 1$, we have that $d(x, y) = \lambda^{n\left(x, y\right)} \ \ \forall x, y \in \Omega$ is also a metric over $\Omega$, with $n(x, y) = \min\left\{k \vert x_k \neq y_k\right\}$.
Moreover, $d$ and $\widetilde{d}$ define the same topology.
The open balls are, $\forall x \in \Omega, \ \ r > 0$
\begin{equation*}
    \mathcal{B}\left(x, r\right) =  \left\{y \in \Omega \ \vert \ d(x, y) \leq r\right\} = \left\{y \in \Omega \ \vert \ x_k = y_k \ \forall \ 1 \leq k \leq \frac{\ln r}{\ln \lambda}\right\}
\end{equation*}
\begin{definition}
    $\mathcal{F}$ is a \textbf{Borel $\sigma$-algebra} if is a set of subsets of $\Omega$ such that $\Omega \in \mathcal{F}$
\end{definition}
So a $\sigma$-algebra is actually a collection of all measurable sets.
\begin{definition}
    The \textbf{space of probability measures} is $\mathcal{P}\left(\Omega\right) = \left\{\mu \ \vert \ \mu\left(\Omega\right) = 1\right\}$
\end{definition}
Taking now a map $T : \Omega \to \Omega$ with a probability measure $\mu$, $\mu$ is called \textbf{T-invariant} if $\forall A \in \mathcal{F}, \mu\left(T^{-1}\left(A\right)\right) = \mu\left(A\right)$.
In other words, the push-forward measure as to satisfy $T * \mu \equiv \mu \cdot T^{-1} = \mu$.
We will denote the space of invariant probability measures with $\mathcal{P}_\mathcal{F} \subset \mathcal{P}$.

The \textbf{shift} map $\sigma : \Omega \to \Omega$ such that $\sigma\left(x_0, x_1, \ldots, x_n, \ldots\right) = \sigma\left(x_1, \ldots, x_n, \ldots\right)$ represents an important map over our space.
Notice that the shift map could represent our time, helping us to code a dynamical system (more details in the next sections).
Moreover, one can verify that if $\mu$ is i.i.d. then it's shift-invariant, i.e. $\mu\left(\sigma^{-1}\left(\left[x_1^n\right]\right)\right) = \mu\left(\left[x_1^n\right]\right)$ and $\mu\left(\left[x_{n+k}^{m+k}\right]\right) = \mu\left(\left[x_n^m\right]\right)$.

\begin{theorem}[Borel-Cantelli lemma]
    \begin{equation*}
        \left\{E_n\right\} \ \vert \ \sum_{n=1}^\infty \mu\left(E_n\right) < \infty \ \implies \mu\left(\lim \sup E_n\right) = 0
    \end{equation*}
\end{theorem}
This lemma states that, beside null-measure sets, typically an $x \in \Omega$ only belongs to finitely many $E_k$'s. \\
In order to clarify, let's take $E_m \subset \Omega$ such that $x \in \lim \sup E_n \Leftrightarrow x \in \cup_{m=n}^\infty E_m \ \forall n$, then it exists $n_j \to \infty$ sequence such that $x \in E_{n_j} \ \forall j$.
Consequently, $E_n$ are becoming rare as $n$ increases.

\begin{definition}
    Let's consider a sequence of events $\left\{E_1, \ldots, E_n\right\}$.
    These are \textbf{independent} if $\mu\left(E_1 \cap \ldots \cap E_n\right) = \mu\left(E_1\right) \ldots \mu\left(E_n\right)$
\end{definition}
\pagebreak

\section{Stochastic Processes}
\begin{definition}
    A \textbf{stochastic process} is an infinite sequence of random variables $X_n$ with values in $\mathcal{A}$ defined by the $k^\text{th}$ order joint distribution:
    \begin{equation*}
        \mu_k\left(a_1^k\right) = \mathbb{P}\left(X_1^k = a_1^k\right) \ \ a_1^k \in \mathcal{A}
    \end{equation*}
\end{definition}
We need also a consistency condition:
\begin{equation*}
    \mu_t\left(a_1^t\right) = \sum_{a_0 \in \mathcal{A}} \mu_{t+1}\left(a_0^t\right) = \sum_{a_{t+1} \in \mathcal{A}} \mu_{t+1}\left(a_1^{t+1}\right)
\end{equation*}
Equivalently, we can define a stochastic process through the conditional probability
\begin{equation*}
    \mu\left(a_t \vert a_1^{t-1}\right) = \frac{\mu_t\left(a_1^t\right)}{\mu_{t-1}\left(a_1^{t-1}\right)}
\end{equation*}
The $\mu_k$ are called \textbf{marginals} and, in order to be a probability, they must satisfy the normalization condition
\begin{equation*}
    \sum_{a_1^k \in \mathcal{A}} \mu_k\left(a_1^k\right) = 1
\end{equation*}
We notice that this sum is exponentially growing in $k$, so it's impossible to approximate the measure.
\begin{definition}
    A stochastic process is \textbf{stationary} if
    \begin{equation*}
        \mu\left(a_1^k\right) = \mu\left(a_{t+1}^{t+k}\right) \ \ \forall a_1^\infty \in \mathcal{A}^\mathbb{N}
    \end{equation*}
\end{definition}
\begin{definition}
    An \textbf{information source} is a stationary, ergodic, stochastic process.
\end{definition}
\begin{definition}
    A process or a source is a \textbf{shift-invariant Borel probability measure} $\mu$ on the topological space $\mathcal{A}^\mathbb{Z}$ of doubly-infinite sequences $x = \left\{x_n\right\}_{n \in \mathbb{Z}}$, drawn from a finite (i.e. countable) alphabet $\mathcal{A}$
\end{definition}
Furthermore, it is trivial that we can write any standard cylinder as
\begin{equation*}
    \left[x_1^t\right] = \sqcup_{a \in \mathcal{A}} \left[x_1, \ldots, x_t, a\right]
\end{equation*}
It's easy to check that
\begin{equation*}
    \mu \in \mathcal{P}_I\left({\Omega}\right) \ \vert \ \mu \circ \sigma^{-1} = \mu \Leftrightarrow \sum_{a \in \mathcal{A}} \mu_{t+1}\left(a, x_1, \ldots, x_t\right) = \mu_t\left(x_1^t\right)
\end{equation*}
Neural networks are heuristically approximating $\mu$.

\begin{theorem}[Kolmogorov representation theorem]
    If $\left\{\mu_n\right\}$ is a sequence of measure defining a process then there is a unique Borel probability measure $\mu$ on $\mathcal{A}^\infty$ such that, $\forall k \geq 1$ and $\forall \left[a_1^k\right]$ cylinder
    \begin{equation*}
        \mu\left(\left[a_1^k\right]\right) = \mu_k\left(a_1^k\right)
    \end{equation*}
\end{theorem}

\subsection{Markov's Models}
Markov's model is a stochastic model used to model pseudo-randomly changing systems.
In a Markov's process the $n$ element probability depends only on previous $k$-elements
\begin{equation}
\mu\left(x_{n}\ \vert\ x_{0}, x_{1}, \dots, x_{n-1}\right):=\mu\left(x_{n}\ \vert\ x_{k}, x_{k+1}, \dots, x_{n-1}\right)
\end{equation}

A Markov's chain is a Markov's process where the $n$ element depends only on the current state ($n-1$ element).
For this reason a Markov's chain is a no memory process.
\begin{equation}
\mu\left(x_{n}\ \vert\ x_{0}, x_{1}, \dots, x_{n-1}\right):=\mu\left(x_{n}\ \vert\ x_{n-1}\right)
\end{equation}

We can define a \emph{Markov's measure}.
Let's call $\mathbf{p}=\left(p_{1},\ p_{2},\dots ,\ p_{l}\right)$ the probability vector that a character of the alphabet $\mathcal{A}$ is extracted and $P=\left[p_{ij}\right]$ the $l\times l$ matrix that describe the probability than the $j$ character is extracted when the privious one is the $i$.
We know that $\mathbf{p}$ is normalized and $P$ is a stochastic matrix
\[p_{j}\geq 0\quad \sum_{j=1}^{l}p_{j}=1\qquad\sum_{h=1}^{j}p_{ij}=1\ \forall i\]
Since $P$ is stochastic it has the uniti vector as eighenvector with $1$ as eighenvalue.
So for the \emph{Person-Frobenius theorem} all the eighenvalues are contained inside the complex circle with radius $1$.
We say that $\mathbf{p}$ is \emph{invariant} if it's a P's eighenvector. We can define for all $n$ the \emph{Markov's measure}
\begin{equation}
\mu_{n}\left(x_{1},\ \dots\ ,\ x_{n}\right)=p_{x_{1}}P_{x_{1}x_{2}}P_{x_{2}x_{3}}P_{x_{n-1}x_{n}}
\end{equation}


\subsection{Hidden Markov's Models}
\pagebreak

\end{document}
